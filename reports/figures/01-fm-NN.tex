\documentclass{article}

\usepackage{tikz}
\begin{document}

%\pagestyle{empty}
%
%\def\layersep{2.5cm}
%
%\begin{tikzpicture}[shorten >=1pt,->,draw=black!50, node distance=\layersep]
%    \tikzstyle{every pin edge}=[<-,shorten <=1pt]
%    \tikzstyle{neuron}=[circle,fill=black!25,minimum size=17pt,inner sep=0pt]
%    \tikzstyle{input neuron}=[neuron, fill=green!50];
%    \tikzstyle{output neuron}=[neuron, fill=red!50];
%    \tikzstyle{hidden neuron}=[neuron, fill=blue!50];
%    \tikzstyle{annot} = [text width=4em, text centered]
%
%
%    % Draw the input layer nodes
%%    \foreach \name / \y in {1,...,4}
%    \foreach \name / \y in {3}
%    % This is the same as writing \foreach \name / \y in {1/1,2/2,3/3,4/4}
%        \node[input neuron, pin=left:Input ] (I-\name) at (0,-\y-0.5) {};
%
%    % Draw the hidden layer nodes
%    \foreach \name / \y in {1,...,7}
%        \path[yshift=0.5cm]
%            node[hidden neuron] (H-\name) at (\layersep,-\y cm) {};
%
%    % Draw the output layer node
%    \node[output neuron,pin={[pin edge={->}]right:Output}, right of=H-4] (O) {};
%
%    % Connect every node in the input layer with every node in the
%    % hidden layer.
%    \foreach \source in {3}
%        \foreach \dest in {1,...,7}
%            \path (I-\source) edge (H-\dest);
%
%    % Connect every node in the hidden layer with the output layer
%    \foreach \source in {1,...,7}
%        \path (H-\source) edge (O);
%
%    % Annotate the layers
%    \node[annot,above of=H-1, node distance=1cm] (hl) {Hidden layer};
%    \node[annot,left of=hl] {Input layer};
%    \node[annot,right of=hl] {Output layer};
%\end{tikzpicture}
%% End of code
\def\input_layer{1}
\def\hidden_layer_0{5}

\def\hdist{2}

\begin{tikzpicture}

% plot the input layer
\foreach \i in {1,...,\input_layer}{
\node (left_input\i) at (-2, {\i*\hdist-(\input_layer + 1)*\hdist/2}) {};
\pgfmathsetmacro\j{int(\i -1)}
\node[draw, circle, minimum size=30pt] (input\i) at (0, {\i*\hdist-(\input_layer + 1)*\hdist/2}) {$x_{0}$};
\draw[->,>=latex] (left_input\i) -- (input\i);
}

% plot the first hidden layer
\foreach \i in {1,...,\hidden_layer_0}{
\pgfmathsetmacro\j{int(\i -1)}
\node[draw, circle, minimum size=30pt] (hidden0\i) at (4, {\i*\hdist-(\hidden_layer_0 + 1)*\hdist/2}) {$x_{1,\j}$};
    \foreach \k in {1,...,\input_layer}{
    \draw[->,>=latex] (input\k) -- (hidden0\i);
    };
}

% plot the second layer
\node[draw,circle, minimum size=30pt] (output) at (8,0) {$x_2$};
\foreach \i in {1,...,\hidden_layer_0}{
\draw[->, >=latex] (hidden0\i) -- (output);
}

%plot sigmoid
\node[draw , circle , minimum size=30pt] (sigmoid) at (10,0) {$x_3$};
\draw[->, >=latex] (output) -- (sigmoid);

\node (outter) at (12,0) {};
\draw[->, >=latex] (sigmoid) -- (outter);



%
%
%
%\node (input) at (-2,0) {} ;
%\node[draw, circle] (n00) at (0,0) {$x_{0,0}$};
%
%\foreach \i in {0,...,\hidden_layer_0}
%  \node[draw, circle] ({\i}) at (2,2*\i)  {};
%  
%\draw[->] (1) -- (2);
%
%\node[draw, circle] (n10) at (4,4) {$x_{1,0}$};
%
%\draw[->, >=latex] (input) -- (n00);
%
%\draw[->,>=latex] (n00) -- (n10);


\end{tikzpicture}



\end{document}