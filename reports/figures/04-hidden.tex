\documentclass{article}

\usepackage{tikz}
\usetikzlibrary{automata,positioning}

\usepackage{graphicx}
%\usepackage{landscape}
 \usepackage{nopageno}

\begin{document}

\rotatebox{90}{
\begin{tikzpicture}

\def\input_layer{1}
\def\hidden_layer_0{3}
\def\hdist{3}

\pgfmathsetmacro\x{0}

% plot the input layer
\foreach \i in {1,...,\input_layer}{
\node (left_input\i) at (\x, {\i*\hdist-(\input_layer + 1)*\hdist/2}) {};
\pgfmathsetmacro\j{int(\i -1)}
\node[draw, circle, minimum size=50pt] (input\i) at (\x+2, {\i*\hdist-(\input_layer + 1)*\hdist/2}) {$x$};
\draw[->,>=latex] (left_input\i) -- (input\i);
}

\pgfmathsetmacro\x{\x+2}
\pgfmathsetmacro\x{\x+4}

% plot the first hidden layer
\foreach \i in {1,...,\hidden_layer_0}{
\pgfmathsetmacro\j{int(\i -1)}
\node[draw, circle, minimum size=50pt, align=center] (hidden0\i) at (\x, {\i*\hdist-(\hidden_layer_0 + 1)*\hdist/2}) {$x_\i :=$ \\ $w_\i x + b_\i$};
    \foreach \k in {1,...,\input_layer}{
    \draw[->,>=latex] (input\k) -- (hidden0\i);
    };
}

\pgfmathsetmacro\x{\x+4}

% plot the sigmoid part of the first hidden layer
\foreach \i in {1,...,\hidden_layer_0}{
\node[draw, circle, minimum size=50pt, align=center] (hidden_sigmoid\i) at (\x, {\i*\hdist-(\hidden_layer_0 + 1)*\hdist/2}) {$y_\i :=$ \\ $\sigma(x_\i)$};
\draw[->,>=latex] (hidden0\i) -- (hidden_sigmoid\i);
}

\pgfmathsetmacro\x{\x+4}

% plot the second layer
\node[draw,circle, minimum size=50pt, align = center] (output) at (\x,0) {$y :=$ \\ $\sum_{i=1}^5 W_i y_i + B $};
\foreach \i in {1,...,\hidden_layer_0}{
\draw[->, >=latex] (hidden_sigmoid\i) -- (output);
}

\pgfmathsetmacro\x{\x+4}

%plot sigmoid
\node[draw , circle , minimum size=50pt] (sigmoid) at (\x,0) {$\sigma(y)$};
\draw[->, >=latex] (output) -- (sigmoid);

\node (outter) at (\x+2,0) {};
\draw[->, >=latex] (sigmoid) -- (outter);

\end{tikzpicture}

}

\newpage

% neural netwark with no arthmetic/algebraic expressions in the nodes

\def\input_layer{1}
\def\hidden_layer_0{4}
\def\hdist{3}

\rotatebox{90}{
\begin{tikzpicture}

\pgfmathsetmacro\x{0}

% plot the input layer
\foreach \i in {1,...,\input_layer}{
\node (left_input\i) at (\x, {\i*\hdist-(\input_layer + 1)*\hdist/2}) {};
\node[draw, circle, minimum size=50pt] (input\i) at (\x+2, {\i*\hdist-(\input_layer + 1)*\hdist/2}) {};
\draw[->,>=latex] (left_input\i) -- (input\i);
}

\pgfmathsetmacro\x{\x+5}

% plot the first hidden layer
\foreach \i in {1,...,\hidden_layer_0}{
%\pgfmathsetmacro\j{int(\i -1)}
\node[draw, circle, minimum size=50pt, align=center] (hidden0\i) at (\x, {\i*\hdist-(\hidden_layer_0 + 1)*\hdist/2}) {};
    \foreach \k in {1,...,\input_layer}{
    \draw[->,>=latex] (input\k) -- (hidden0\i);
    };
}

\pgfmathsetmacro\x{\x+2.5}

% plot the sigmoid part of the first hidden layer
\foreach \i in {1,...,\hidden_layer_0}{
\node[draw, circle, minimum size=50pt, align=center , draw=blue] (hidden_sigmoid\i) at (\x, {\i*\hdist-(\hidden_layer_0 + 1)*\hdist/2}) {};
\draw[->,>=latex] (hidden0\i) -- (hidden_sigmoid\i);
}

\pgfmathsetmacro\x{\x+3}

% plot the second layer
\node[draw,circle, minimum size=50pt, align = center] (output) at (\x,0) {};
\foreach \i in {1,...,\hidden_layer_0}{
\draw[->, >=latex] (hidden_sigmoid\i) -- (output);
}

\pgfmathsetmacro\x{\x+3}

%plot sigmoid
\node[draw , circle , minimum size=50pt , draw = blue] (sigmoid) at (\x,0) {};
\draw[->, >=latex] (output) -- (sigmoid);

\node (outter) at (\x+2,0) {};
\draw[->, >=latex] (sigmoid) -- (outter);

\end{tikzpicture}

}

\end{document}